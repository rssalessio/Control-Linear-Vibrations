\documentclass[10pt,a4paper]{article}
\usepackage[utf8]{inputenc}
\usepackage{amsmath}
\usepackage{amsfonts}
\usepackage{amssymb}
\author{Alessio Russo}
\title{Test1}
\begin{document}
\tableofcontents

\section{Introduzione}
Rotazione del disco attaccato al motore: $\theta_m$. \\
Raggio del disco attaccato al motore: $R_d$. \\
Posizione del carretto : $x$. \\
Tensione in ingresso al motore: $V$ \\
Stiffness molla: $K$ \\
Resistenza e induttanza del motore: $R,L$ \\
Massa carretto+peso: $M$ \\
Costante torque/backemf: $K_e$

Essendo il gearbox fissato al carretto abbiamo:
$$\theta_m = \frac{x}{R_d} \Rightarrow \dot{\theta}_m = \frac{\ddot{x}}{R_d}$$

Funzione di trasferimento tra forza erogata e posizione del carretto:
$$\frac{X}{F}(s) = \frac{1}{M} \frac{1}{s^2+\frac{K}{M}}$$

Funzione di trasferimento tra Tensione e corrente:
$$\frac{I}{V}(s) = \frac{s^2+\frac{K}{M}}{(s^2+\frac{K}{M})(2R+2sL)+\gamma s}$$
$$\gamma = \frac{4K_e^2}{R_d^2M}$$
Per $\gamma \ll 1$:
$$\frac{I}{V} \approx \frac{1}{2R+2sL}$$
\newpage
\section{Ricavare $k$}
Per ricavare le $k$ delle molle l'idea generale è di guardare il displacement $x$ del carretto applicando la stessa forza $F(t)$. \\ \\
Per $F(t)=F_0$ abbiamo $x(t) \to kF_0$. \\
Chiamiamo le due molle $k_1,k_2$. Noi non conosciamo $F_0$, ma è costante per entrambi e cambia solo il $k$. Quindi:
$$F_0 = k_1 x_1$$
$$F_0 = k_2 x_2$$
Quindi prendendo il rapporto:

$$\frac{x_1}{x_2} = \frac{k_2}{k_1}$$
Se $x_2 < x_1$ allora $k_2 > k_1$, e viceversa.  Da questa formula possiamo ricavare il rapporto fra le due $k$ e verificare con quelle scritte nel manuale. \\ \\
Quindi:
\begin{enumerate}
\item Aprire il file simulink testK nella cartella tests/15March.
\item Caricare lo schema sulla scheda
\item Attaccare la prima molla
\item Far andare la scheda, calcolare $x_1$ quando il carretto è fermo
\item Spegnere la scheda, staccare la molla e attaccare l'altra molla
\item Far andare la scheda, calcolare $x_2$ e prendere il rapporto
\end{enumerate}
\newpage
\section{Ricavare i parametri del motore}
\end{document}