\documentclass[10pt,a4paper]{article}
\usepackage[utf8]{inputenc}
\usepackage{amsmath}
\usepackage{amsfonts}
\usepackage{amssymb}
\author{Alessio Russo}
\title{Test1}
\begin{document}
\tableofcontents

\section{Introduzione}
Rotazione del disco attaccato al motore: $\theta$. \\
Raggio del disco attaccato al motore: $0.5D$. \\
Posizione del carretto : $x$. \\
Tensione in ingresso al motore: $v$ \\
Stiffness,damping molla: $K,C$ \\
Resistenza e induttanza del motore: $R,L$ \\
Massa carretto+peso e inrzia disco: $M=M_1+ 2\frac{J}{D}$ \\
Costante torque/backemf: $K_e$

Essendo il gearbox fissato al carretto abbiamo:
$$\theta= \frac{x}{R_d} $$

Funzione di trasferimento tra forza erogata e posizione del carretto:
$$\frac{X}{F}(s) =  \frac{1}{Ms^2+Cs +K}$$

Funzione di trasferimento tra Tensione/Backemf e forza:
$$F(s) = 4\frac{K_e}{D}\frac{1}{2R+2sL}(V(s)-4\frac{K_e}{D} sX(s))$$
Tra torque ecorrente abbiamo:
$$C(s) = 2K_e I(S)$$
e $$F(s) = \frac{2}{D}C(s)$$
Quindi fra tensione e corrente abbiamo:
$$\frac{I}{V}(s) = \frac{Ms^2+Cs+K}{(Ms^2+Cs+K)(2R+2sL)+\gamma s } =M(s)$$
$$\gamma = \Big(\frac{4K_e}{D}\Big)^2$$
Sviluppando il termine al denominatore otteniamo:
$$2MLs^3 + s^2(2MR+2CL) +s(2CR+2KL+\gamma ) + 2KR$$

\newpage
\section{Ricavare $k$}
Per ricavare le $k$ delle molle l'idea generale è di guardare il displacement $x$ del carretto applicando la stessa forza $F(t)$. \\ \\
Per $F(t)=F_0$ abbiamo $x(t) \to kF_0$. \\
Chiamiamo le due molle $k_1,k_2$. Noi non conosciamo $F_0$, ma è costante per entrambi e cambia solo il $k$. Quindi:
$$F_0 = k_1 x_1$$
$$F_0 = k_2 x_2$$
Quindi prendendo il rapporto:

$$\frac{x_1}{x_2} = \frac{k_2}{k_1}$$
Se $x_2 < x_1$ allora $k_2 > k_1$, e viceversa.  Da questa formula possiamo ricavare il rapporto fra le due $k$ e verificare con quelle scritte nel manuale. \\ \\
Quindi:
\begin{enumerate}
\item Aprire il file simulink test1 nella cartella tests/15March.
\item Caricare lo schema sulla scheda
\item Attaccare la prima molla
\item Far andare la scheda, calcolare $x_1$ quando il carretto è fermo
\item Spegnere la scheda, staccare la molla e attaccare l'altra molla
\item Far andare la scheda, calcolare $x_2$ e prendere il rapporto
\end{enumerate}
\newpage
\section{Stimare la massa/Inerzia}
Consideriamo la funzione di trasferimento fra $F(s)$ e $X(s)$:
$$X(s)= \frac{1}{M}\frac{1}{s^2+\frac{K}{M}}F(s)$$
Dove $F(s) = \frac{N}{D}(s)$. Supponiamo $F$ quasi costante, quindi approssimabile a $\frac{1}{s}$. Per i fratti semplici otteniamo:
$$(x) = \frac{As+B}{s^2+\frac{K}{M}}+\cdots$$
Dove $\cdots$ intende fratti dipendenti $F(s)$. Sappiamo che il primo fratto è il termine del transitorio, mentre $\cdots$  indica i termini a regime che dovrebbe essere approssimabile a $\frac{C}{s}$. \\ \\
Quindi nel transitorio abbiamo una sinusoide (smorzata perchè comunque lo smorzamento c'è anche se non modellato, ma non influisce sulla frequenza di oscillazione) con pulsazione $\omega = \sqrt{\frac{K}{M}}$. Quindi esegui il test1 con $V=3.6$ costante, guarda l'encoder e misura il periodo fra le due sinusoidi:
$$\frac{2\pi}{T}= \sqrt{\frac{K}{M}} \Rightarrow M = \frac{KT^2}{4\pi^2}$$
Dovrebbe venire $M > 0.5kg$.
Inoltre sapppiamo che $M$ è uguale alla massa del load + $2\frac{J}{D}$. Quindi sapendo la massa del carretto, la massa del load è $0.5$+ massa del carretto, sappiamo $D$ e ricaviamo $J$.
\newpage
\section{Ricavare $K_e$ e stimare $\gamma$}
Quindi attacca ora una delle due molle, tipo $k_1$.Sai che eseguendo il test1 otterai che il carretto è a una posizione $x_1$. Quindi $F=k_1 x_1$.
Conoscendo ora la forza, e sapendo $D$, sappiamo che il torque è dato da:

$$T = F\frac{D}{2} = 2K_e i$$
Possiamo misurare la corrente del motore attraverso l'output current del blocco del motore su simulink.

Rieseguiamo l'esperimento test1, e salviamo l'output della corrente in una variabile di matlab $i$ (gia' fatto nello shcema).

A regime $F=F_0$. Prendiamo l'ultimo sample della corrente, che sarà $i(N)$. Nota che su matlab la corrente viene salvata nella struttura $i$ . per accedere ai dati devi fare $i.Data$.

 Quindi:
$$K_e = \frac{FD}{4i_N}$$

Per stimare $\gamma$ sappiamo che vale:
$$\gamma = \Big(\frac{4K_e}{D}\Big)^2$$
Sapendo $K_e$ puoi calcolarla.

\newpage
\section{Ricavare la resistenza del motore}
Per ricavare il guadagno del motore conviene sfruttare il fatto che possiamo misurare la corrente del motore attraverso l'output current del blocco del motore su simulink. \\ \\
Considerando la back-emf la funzione di trasferimento è data da:
$$\frac{I}{V}(s) = M(s)$$
Per $V=V_0$ (nota che nello schema $V_0=3.6$) costante otteniamo:
$$\lim_{t \to \infty} i(t) = \lim_{s \to 0} sI(s)\frac{3.6}{s} = \frac{3.6}{2R}$$
Quindi prendiamo l'ultimo campione della corrente (che chiamo $i_N$), e di conseguenza:
$$R = \frac{3.6}{2i_N}$$
Quindi esegui le seguenti istruzioni (devi aver ricavato gamma):
\begin{enumerate}
\item Esegui il test 1
\item $i=i.Data;$
\item $N=length(i);$
\item $ R = (3.6/(2*i(N));$
\end{enumerate}
\section{Ricavare l'induttanza del motore e la costante di tempo, whitebox identification}
$$\frac{I}{V}(s) = \frac{Ms^2+Cs+K}{(Ms^2+Cs+K)(2R+2sL)+\gamma s } =M(s)$$
$$\gamma = \Big(\frac{4K_e}{D}\Big)^2$$
Il sistema $\forall \gamma > 0$ è un passabasso. L'effetto di $\gamma$ è di spostare due poli lontano dai due zeri,con modulo $\to \infty$ Questi due poli rimangono complessi coniugati, stessa cosa per gli zeri. Facendo quindi un diagramma di bode al variare di $\gamma$ si nota che rimane passabasso il sistema in generale. Notiamo che il polo del motore, $\frac{R}{L} \gg 1$, quindi la risposta nel tempo è aprossimabile a:
$$i(t) = \frac{V_0}{2R}(1-e^{-\frac{R}{L}t})$$
Dove $V_0$ è la costante dello step in ingresso che è $V_0=3.6$
Sappiamo che per:
$$\frac{R}{L}t > 6 \Rightarrow e^{-\frac{R}{L}t} < \frac{1}{400}$$
Quindi :
$$\frac{V_0}{2R} - i(t) = \frac{V_0}{2R}e^{-\frac{R}{L}t}$$
Bisogna vedere quadno:
$$\frac{V_0}{2R}e^{-\frac{R}{L}t}= \frac{V_0}{2R} - i(t) < \frac{V_0}{2R}e^{-6}$$
Che equivale alla condizione:
$$1- \frac{i(t)2R}{V0} < e^{-6}$$
Una volta ricavato per quale primo valore di $t$ vale, allora chiamato $t_0$ questo valore abbiamo:
$$\frac{R}{L}t_0 \approx 6 \Rightarrow L = \frac{R}{6}t_0$$\\ \\

Tutto questo lo fa uno script che è nella stessa cartella dove ho salvato test1. Esegui quindi test1, salva corrente e input in due variabili x,u. Salvati il tempo pure in t ed esegui lo script.

Lo script eseguirà identificazione pure di un sistema G1 con 1 polo, G2 con 3 poli/2 zeri.

\end{document}