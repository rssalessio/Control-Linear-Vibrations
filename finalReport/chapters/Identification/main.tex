\chapter{System Identification}
The system considered can be easily modelled and identified without the need to use black-box identification to identify the system. \\ For completeness both \emph{white-box} and \emph{grey-box} identification are used.\\ \\
First of all the problem of whether to consider a \emph{closed} or \emph{open} loop system is considered. In fact \emph{back-emf} can be seen as a gain acting on the velocity of the cart, thus it's a gain on the closed loop.\\ \\
Then, using both \emph{white-box} and \emph{grey-box} identification we identified the main parameters of the system: 
\begin{enumerate}
\item Resistance and inductance for the motor.
\item Mass, stiffness and damping for the cart and the springs.
\end{enumerate}
Last, identification of non-linearities are considered.
\subsection{Validation cost function}

Parlare della cost function usata (NMSE), confrontare con L2 e NRMSE.
Dettagli su  https://rem.jrc.ec.europa.eu/RemWeb/atmes2/20b.htm
Comando:goodnessOfFit(y,x,'NMSE');
\subimport{common}{ol_vs_cl_identification}
\newpage
\subimport{common}{white_box_identification}
\newpage
\subimport{common}{gray_box_identification}
\newpage
\subimport{common}{non_linearities_identification}