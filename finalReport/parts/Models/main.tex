
\part{System modelling}

\chapter{Motor, Pinion and Rack modelling} 
A DC Brushless motor can be modelled with a simple low-pass filter transfer function. Because of that, only a resistance $R$ and an inductance $L$ are needed to model it. Thus, let $D$ be the diameter of the disk attached to the motor, $\theta$ the the angular position of the disk, $J$ the inertia of the motor, and $c_{l}(t)$ the load torque. We can also assume nonlinearities based on the angle and its rate $\dot{\theta}$:  we will denote such nonlinearities with $f_{m}(\theta,\dot{\theta})$.\\ \\
 The output torque of the motor, then, is  given by:
$$c(s) = \frac{K_{e}}{Ls+R} (v(s)-K_{e}s \theta(s))$$
where $v(s)$ is the Laplace transform of $v(t)$, the input voltage to the motor.\\ \\
The differential equation describing the motion of the disk is:
$$J\ddot{\theta}=c(t)-c_{l}(t) -f_{m}(\theta,\dot{\theta})$$
In our case, since it's not possible to detach the pinion and the rack from the motor, $J$ includes the inertia of that sytem.

\chapter{Pinion and Rack}


\chapter{1 DOF Modelling}

\chapter{2 DOF Modelling}

\chapter{3 DOF Modelling}
Equations of motion:

$$J\ddot{\theta}=c(t)-c_l(t) - f_m(\dot{\theta})$$
$$M\ddot{x}+C\dot{x}+Kx = F(t) - f_c(\dot{x})$$
$$\frac{D}{2}\theta = x$$
$f_m$ describes the viscous friction of the motor, $f_c$ describes the friction of the cart. The gearbox is assumed ideal. \\ \\
Therefore $F(t)$ is the transmitted linear force from the motor, thus:
$$F(t)\frac{D}{2} = c_l(t) \Rightarrow F(t) = \frac{2}{D} \Big(c(t)-J\ddot{\theta}-f_m(\dot{\theta})\Big)$$
In the end we obtain:
$$\Big(M+\frac{4}{D^2} J\Big) \ddot{x}+C\dot{x}+Kx = \frac{2}{D}c(t) -\frac{2}{D}f_m(\dot{\theta}) - f_c(\dot{x})$$
In case the gearbox is not assumed ideal, we have:
$$J\ddot{\theta}= \begin{cases}
c(t)-c_l(t) - f_m(\dot{\theta}) \quad \text{ in contact } \\
c(t)-f_m(\dot{\theta}) \quad \text{ not in contact }
\end{cases}
$$
And
$$F(t) = \begin{cases}
\frac{2}{D}c_l(t)\quad \text{ in contact } \\
0 \quad \text{ otherwise }
\end{cases}
$$

\section{Model1 - no BEMF,no disk inertia, no friction cart, no friction motor, no backlash}
$$M\ddot{x} + C\dot{x}+Kx = 2\frac{c(t)}{D}, \quad \theta = \frac{2}{D}x$$
$$\mathcal{L}\{c(t)\} = 2K_e \frac{1}{2R+2sL} \mathcal{L}\{ v(t)\}$$

\section{Model2 - no friction cart, no friction motor, no backlash}
$$M\ddot{x} + C\dot{x}+Kx = 2\frac{c(t)}{D}  - 4\frac{J}{D^2}\ddot{x}, \quad \theta = \frac{2}{D}x$$
$$\mathcal{L}\{c(t)\} = 2K_e \frac{1}{2R+2sL} (\mathcal{L}\{ v(t)\}-2K_e s \mathcal{L}\{\theta \})$$


\section{Model 3 - no friction motor, backlash}
$$M\ddot{x} + C\dot{x}+Kx = 2\frac{c(t)}{D} - 4\frac{J}{D^2}\ddot{x} - f_c(\dot{x}), \quad \theta = \frac{2}{D}x$$
$$\mathcal{L}\{c(t)\} = 2K_e \frac{1}{2R+2sL} (\mathcal{L}\{ v(t)\}-2K_e s \mathcal{L}\{\theta \})$$



\section{Model 4 }
$$M\ddot{x} + C\dot{x}+Kx = F(t) - 4\frac{J}{D^2}\ddot{x}- f_c(\dot{x})$$
$$\mathcal{L}\{c(t)\} = 2K_e \frac{1}{2R+2sL} (\mathcal{L}\{ v(t)\}-2K_e s \mathcal{L}\{\theta \})$$
See introduction for gearbox modelling.
\begin{enumerate}
\item $\mathcal{L}\{\cdot \}$ Laplace transform.
\item $J$ Disk inertia.
\item $M$ Cart+load mass
\item $C$ Spring damping.
\item $K$ Spring stiffness.
\item $c(t)$ Torque.
\item $D$ Disk diameter.
\item $f_c(t)$ friction applied to the cart.
\item $f_g(t)$ sliding friction applied to the teeth between the gearbox and the disk.
\item $f_m$ friction of the motor
\item $\theta$ angle of the disk.
\item $v(t)$ tension applied to the motor.
\item $R,L$ resistance and inductance of the motor
\item $K_e$ backemf constant.

\end{enumerate}

\section{2 DOF - Model}
To derive the equations of motion we can use the Lagrangian approach. Let $T,V,D$ be the kinetic, potential and dissipated energy. Then:
\begin{align*}
T  &= \frac{1}{2} \Big(M_1 + \frac{4}{D^2}J \Big) \dot{x_1}^2 + \frac{1}{2}M_2 \dot{x_2}^2 \\
V &= \frac{1}{2}k_1x_1^2 + \frac{1}{2}k_2(x_2-x_1)^2 \\
D &= \frac{1}{2}c_1\dot{x_1}^2 + \frac{1}{2}c_2(\dot{x_2}-\dot{x_1})^2
\end{align*}
Let $Q$ be the external forces acting on the systems:
\begin{align*}
Q_1 &= \frac{2}{D}c(t) - \frac{2}{D}f_m(\dot{\theta}) - f_c(\dot{x_1}) \\
Q_2 &=  - f_c(\dot{x_2})
\end{align*}
The equations of motion are given by:
$$\frac{d}{dt}\Big(\frac{\partial T}{\partial x_i} \Big) -\frac{\partial T}{\partial \dot{x}_i} + \frac{\partial V}{\partial x_i} + \frac{\partial D}{\partial \dot{x}_i} = Q_i$$
\begin{align*}
\Big(M_1+\frac{4}{D^2}J \Big)\ddot{x}_1+(c_1+c_2)\dot{x}_1 +(k_1+k_2)x_1 &= k_2 x_2 +c_2 \dot{x}_2 + \frac{2}{D}(c(t)-f_m(\dot{\theta}))-f_c(\dot{x}_1) \\
M_2 \ddot{x}_2 +c_2 \dot{x}_2 +k_2x_2 &= k_2x_1+c_2\dot{x}_1 -f_c(\dot{x}_2)
\end{align*}
Thus we can write:
$$
\begin{bmatrix}
\hat{M}_1 & 0 \\
0 & M_2
\end{bmatrix}
\ddot{x} + 
\begin{bmatrix}
c_1+c_2 & -c_2 \\
-c_2 & c_2
\end{bmatrix}
\dot{x}+
\begin{bmatrix}
k_1+k_2 & -k_2 \\
-k_2 &k_2
\end{bmatrix}
x =Q(t)$$
Where:
$$Q(t)=\begin{bmatrix}1 \\ 0 \end{bmatrix} F(t)$$
Thus:
$$\ddot{x} = M^{-1}C\dot{x}+M^{-1}Kx+ M^{-1}BF(t)$$
