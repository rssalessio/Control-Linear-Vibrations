
%\part{System modelling}
\chapter{Linear system modelling}
\begin{figure}[!h]
\centering
\begin{tikzpicture}[every node/.style={draw,outer sep=0pt,thick}]
\tikzstyle{spring}=[thick,decorate,decoration={zigzag,pre length=0.3cm,post length=0.3cm,segment length=6}]
\tikzstyle{damper}=[thick,decoration={markings,  
  mark connection node=dmp,
  mark=at position 0.5 with 
  {
    \node (dmp) [thick,inner sep=0pt,transform shape,rotate=-90,minimum width=15pt,minimum height=3pt,draw=none] {};
    \draw [thick] ($(dmp.north east)+(2pt,0)$) -- (dmp.south east) -- (dmp.south west) -- ($(dmp.north west)+(2pt,0)$);
    \draw [thick] ($(dmp.north)+(0,-5pt)$) -- ($(dmp.north)+(0,5pt)$);
  }
}, decorate]
\tikzstyle{ground}=[fill,pattern=north east lines,draw=none,minimum width=0.75cm,minimum height=0.3cm]
\begin{scope}[xshift=7cm]
\node (M) [minimum width=1cm, minimum height=2.5cm] {$M_1$};

\node (ground) [ground,anchor=north,yshift=-0.25cm,minimum width=1.5cm] at (M.south) {};
\draw (ground.north east) -- (ground.north west);
\draw [thick] (M.south west) ++ (0.2cm,-0.125cm) circle (0.125cm)  (M.south east) ++ (-0.2cm,-0.125cm) circle (0.125cm);

\node (wall) [ground, rotate=-90, minimum width=1cm,yshift=-3cm, xshift=-0.5cm] {};
\draw (wall.north east) -- (wall.north west);

\draw [spring] (wall.140pt) -- ($(M.north west)!(wall.140pt)!(M.south west)$) node [draw=none,midway,above=0.3cm] {$k_1$};
\draw [damper] (wall.30pt) -- ($(M.north west)!(wall.30pt)!(M.south west)$) node [draw=none,midway,below=0.3cm] {$c_1$};


\draw[->]  (M.north) -- ++ (0,0.5cm) -- +(0.5cm,0) node [draw=none, midway,above=0.1cm](x1){$x_1$};


\node (M2) [minimum width=1cm, minimum height=2.5cm, right= 2cm of {$(M)!0.5!(M)$}] {$M_2$};
\node (ground2) [ground,anchor=north,yshift=-0.25cm,minimum width=1.5cm] at (M2.south) {};
\draw (ground2.north east) -- (ground2.north west);
\draw [thick] (M2.south west) ++ (0.2cm,-0.125cm) circle (0.125cm)  (M2.south east) ++ (-0.2cm,-0.125cm) circle (0.125cm);

\draw [spring] ($(M.north east)!(wall.140pt)!(M.south east)$) -- ($(M2.north west)!(wall.140pt)!(M2.south west)$) node [draw=none,midway,above=0.3cm] {$k_2$};
\draw [damper] ($(M.north east)!(wall.30pt)!(M.south east)$) -- ($(M2.north west)!(wall.30pt)!(M2.south west)$) node [draw=none,midway,below=0.3cm] {$c_2$};


\node (M3) [minimum width=1cm, minimum height=2.5cm, right= 2cm of {$(M2)!0.5!(M2)$}] {$M_3$};
\node (ground3) [ground,anchor=north,yshift=-0.25cm,minimum width=1.5cm] at (M3.south) {};
\draw (ground3.north east) -- (ground3.north west);
\draw [thick] (M3.south west) ++ (0.2cm,-0.125cm) circle (0.125cm)  (M3.south east) ++ (-0.2cm,-0.125cm) circle (0.125cm);

\draw [spring] ($(M2.north east)!(wall.140pt)!(M2.south east)$) -- ($(M3.north west)!(wall.140pt)!(M3.south west)$) node [draw=none,midway,above=0.3cm] {$k_3$};
\draw [damper] ($(M2.north east)!(wall.30pt)!(M2.south east)$) -- ($(M3.north west)!(wall.30pt)!(M3.south west)$) node [draw=none,midway,below=0.3cm] {$c_3$};

\draw   (-5,0.2) circle[radius=1cm] node (motor)[draw=none] {};
\draw (-5,-0.8) -- ($(M.south west)!0.45cm!(M.north west)$);
\draw[pattern=north west lines]   (-5,0.2) circle[radius=0.1cm] node(shaft)[draw=none] {};
\node (rack) [rectangle, minimum width=2cm, minimum height=0.45cm, anchor=north] at(-5,-0.8) {};
\node (ground4) [ground,anchor=north,yshift=-0.725cm,minimum width=2.5cm] at (-5,-0.8){};
\draw (ground4.north east) -- (ground4.north west);
\draw [thick] (rack.south west) ++ (0.2cm,-0.125cm) circle (0.125cm)  (rack.south east) ++ (-0.2cm,-0.125cm) circle (0.125cm);

\draw [->] (-4,1) arc [start angle=30, end angle=120, radius=1.1cm] node[draw=none,anchor=south,xshift=1cm](angle) {$\theta, c(t)$};

\draw[->]  (M2.north) -- ++ (0,0.5cm) -- +(0.5cm,0) node [draw=none, midway,above=0.1cm](x1){$x_2$};
\draw[->]  (M3.north) -- ++ (0,0.5cm) -- +(0.5cm,0) node [draw=none, midway,above=0.1cm](x1){$x_3$};


\draw[->]  (ground4.south) -- ++ (0,-0.5cm) -- +(0.5cm,0) node [draw=none, midway,right=0.3cm](x1){$x_0$};
\draw[dashed] (rack.south) -- (ground4.south);
\draw (-5,0.2) -- (-4.5,1.066) node[draw=none,midway,anchor=west] (diam){$\frac{D}{2}$};

\end{scope}
\end{tikzpicture}
\end{figure}
\section{Motor, Pinion and Rack modelling} 
A DC Brushless motor can be modelled with a simple low-pass filter transfer function. Because of that, only a resistance $R$ and an inductance $L$ are needed to model it. Thus, let $D$ be the diameter of the disk attached to the motor, $\theta$ the the angular position of the disk, $J$ the inertia of the motor, and $c_{l}(t)$ the load torque. We can also assume non-linearities based on the angle and its rate $\dot{\theta}$:  we will denote such non-linearities with $f_{m}(\theta,\dot{\theta})$.\\ \\
 The output torque of the motor, then, is  given by:
 \begin{equation}
 c(s) = K_e i(s)
 \end{equation}
 where $K_e$ \SI{}{\newton \metre \per \ampere} is the electrical torque constant.
And
\begin{equation}
i(s)=\frac{1}{Ls+R} (v(s)-K_{e}s \theta(s))
\end{equation}
where $v(s)$ is the Laplace transform of $v(t)$, the input voltage to the motor. Notice that the term $K_{e}s \theta(s)$ describes the back-emf effect. \\
The differential equation describing the motion of the disk is:
$$J\ddot{\theta}=c(t)-c_{l}(t) -f_{m}(\theta,\dot{\theta})$$
In our case, since it's not possible to detach the pinion and the rack from the motor, $J$ includes the inertia of that system, and $c_{l}(t)$ is the torque load attached to the rack. 
\\ \\
The rack position is given by $x$, which is equal to $0$ when the rack is at the center. Since the rack is attached to the disk, and by neglecting non-lineart effects such as back-lash, we can say that $x=\frac{D}{2}\theta $. Moreover,  the total force transmitted from the motor is:
$$F(t)  =  \frac{2}{D} (c(t)-J\ddot{\theta}) = \gamma i(t) - \frac{4}{D^2}J \ddot{x}$$
Where $\gamma = \frac{2K_e}{D}$. 

\section{Carts, Springs and Dampers Modelling}
Each cart has the same shape and mass $M_{c}$. Let $M_{i}$ denote the total mass of the $i$-eth cart including the load, $i \in {1,2,3}$. \\
Let $x_{i}$ be the position of each cart, in \SI{}{\cm}. The small friction coefficient of the sliding guide can be approximated with a viscous friction $C_{s}(M)$ which depends on the mass of the cart plus the load. \\ \\
Finally, each spring is modeled as linear spring. Since we have 3 springs, we have labeled their stiffness as $K_l, K_m, K_h$ where $l,m,h$ stand for low, medium and high. The damping contribute given by each spring is labeled as $C_l, C_m, C_h$. Therefore let $C_{i}$ denote the total damping contribution for the $i$-eth cart.\\ \\
Therefore, because of Newton's First Law, each cart has equation:
$$M_i \ddot{x}_i + C_{i} \dot{x}_i + K_{i}x_{i}=F(t)$$
Thus, for each cart 2 states are needed to describe its behaviour.\\ \\
\section{1 DOF State Space Model}
Including the motor 3 states are needed to describe the behaviour of the system. Let $x_{1}$ be the state of the motor (i.e. the current), $x_{2}$ the position of the cart and $x_{3}$ its velocity. The back-emf term in the motor is given by: $-K_e s \theta = -K_e \frac{2}{D}\dot{x} = -\gamma x_3$. Therefore the equations to consider are:
\begin{equation}
\begin{cases}
\dot{x}_1 = -\frac{R}{L}x_{1}+\frac{1}{L} v(t)-\frac{\gamma}{L}x_3  \\
M\dot{x}_3 = -Cx_3 -K x_2 + F(t)
\end{cases}
\end{equation}
Since:
$$F(t)=\gamma i(t) - \frac{4}{D^2}J \ddot{x}$$
By letting $\hat{M} = (M+\frac{4}{D^2}J)$, we have:

\begin{equation}
\begin{cases}
\dot{x}=\begin{bmatrix}
-\frac{R}{L} &0 & -\frac{\gamma}{L} \\
0 & 0 & 1 \\ 
 \frac{\gamma}{\hat{M}} & -\frac{K}{\hat{M}} & -\frac{C}{\hat{M}}
\end{bmatrix}
+
\begin{bmatrix}\frac{1}{L} \\ 0 \\ 0\end{bmatrix} v(t)
\end{cases}
\end{equation}
where $v(t)$ is the external control input.
\section{2 DOF State Space Model}
To derive the equations of motion for the carts we can use the Lagrangian approach.\\ Let $T,V,D$ be the kinetic, potential and dissipated energy. Then:
\begin{align*}
T  &= \frac{1}{2} \Big(M_1 + \frac{4}{D^2}J \Big) \dot{x_1}^2 + \frac{1}{2}M_2 \dot{x_2}^2 \\
V &= \frac{1}{2}k_1x_1^2 + \frac{1}{2}k_2(x_2-x_1)^2 \\
D &= \frac{1}{2}c_1\dot{x_1}^2 + \frac{1}{2}c_2(\dot{x_2}-\dot{x_1})^2
\end{align*}
Let $Q$ be the external forces acting on the systems:
\begin{align*}
Q =\begin{bmatrix} 1 \\ 0 \end{bmatrix} \gamma i(t)
\end{align*}
The equations of motion are given by:
$$\frac{d}{dt}\Big(\frac{\partial T}{\partial x_i} \Big) -\frac{\partial T}{\partial \dot{x}_i} + \frac{\partial V}{\partial x_i} + \frac{\partial D}{\partial \dot{x}_i} = Q_i$$
Thus we can write:
$$
\begin{bmatrix}
\hat{M}_1 & 0 \\
0 & M_2
\end{bmatrix}
\ddot{x} + 
\begin{bmatrix}
c_1+c_2 & -c_2 \\
-c_2 & c_2
\end{bmatrix}
\dot{x}+
\begin{bmatrix}
k_1+k_2 & -k_2 \\
-k_2 & k_2
\end{bmatrix}
x = \begin{bmatrix} 1 \\ 0 \end{bmatrix}\gamma i(t)$$
Then:
$$\ddot{x} = M^{-1}C\dot{x}+M^{-1}Kx+ M^{-1}B\gamma i(t)$$
where:
$$M=\begin{bmatrix}
\hat{M}_1 & 0 \\
0 && M_2
\end{bmatrix}, C=\begin{bmatrix}
c_1+c_2 & -c_2 \\
-c_2 & c_2
\end{bmatrix}, K=\begin{bmatrix}
k_1+k_2 & -k_2 \\
-k_2 & k_2
\end{bmatrix}, B=\begin{bmatrix}1\\ 0 \end{bmatrix}$$
Finally, let $x_1$ be the state of the motor, $x_2$ and $x_3$ the position of the first and second cart, and $x_4$, $x_5$ their velocities, then the stace space model, is:

\begin{equation}
\begin{cases}
\dot{x}=
\left[ \begin{array} { c|c|c  } 
                \begin{array}{c} 
               -\frac{R}{L} \\ 
                0 \\
                0 
                \end{array} &
                \begin{array}{cc} 
               0 & 0 \\ 
                0 & 0 \\
                0  & 0 
                \end{array}&
                \begin{array}{cc} 
               -\frac{\gamma}{L} & 0 \\ 
                1 & 0 \\
                0  & 1 
                \end{array} \\
                \hline 
                M^{-1}B\gamma& -M^{-1}K & -M^{-1}C
\end{array} \right] 
x+\begin{bmatrix}\frac{1}{L} \\ 0 \\ 0 \\ 0 \\ 0\end{bmatrix}v(t)
\end{cases}
\end{equation}
where $x \in \mathbb{R}^5$.

\section{3 DOF State Space Model}
As for 2 degree of freedom we can make use again of the Lagrangian Approach:
\begin{align*}
T &= \frac{1}{2} (M_1 + \frac{4}{D^2}J)\dot{x}_1^2+\frac{1}{2}M_2\dot{x}_2^2 + \frac{1}{2}M_3 \dot{x}_3^2 \\
V &= \frac{1}{2}k_1x_1^2 + \frac{1}{2}k_2(x_2-x_1)^2 +\frac{1}{2}k_3 (x_3-x_2)^2\\
D &= \frac{1}{2}c_1\dot{x_1}^2 + \frac{1}{2}c_2(\dot{x_2}-\dot{x_1})^2 +\frac{1}{2}c_3(\dot{x}_3-\dot{x}_2)^2
\end{align*}

Let $Q$ be the external forces acting on the systems:
\begin{align*}
Q =\begin{bmatrix} 1 \\ 0 \\ 0 \end{bmatrix} \gamma i(t)
\end{align*}
The equations of motion are given by:
$$\frac{d}{dt}\Big(\frac{\partial T}{\partial x_i} \Big) -\frac{\partial T}{\partial \dot{x}_i} + \frac{\partial V}{\partial x_i} + \frac{\partial D}{\partial \dot{x}_i} = Q_i$$
Thus we can write:
$$
\begin{bmatrix}
\hat{M}_1 & 0 & 0 \\
0 & M_2 & 0 \\
0 & 0 & M_3
\end{bmatrix}
\ddot{x} + 
\begin{bmatrix}
c_1+c_2 & -c_2 &0 \\
-c_2 & c_2+c_3 & -c_3 \\
0 & -c_3 & c_3
\end{bmatrix}
\dot{x}+
\begin{bmatrix}
k_1+k_2 & -k_2  &0\\
-k_2 & k_2+k_3 & -k_3 \\
0 & -k_3 & k_3
\end{bmatrix}
x = \begin{bmatrix} 1 \\ 0 \\ 0\end{bmatrix}\gamma i(t)$$
Then:
$$\ddot{x} = M^{-1}C\dot{x}+M^{-1}Kx+ M^{-1}B\gamma i(t)$$
where:
$$M=\begin{bmatrix}
\hat{M}_1 & 0 & 0 \\
0 & M_2 & 0 \\
0 & 0 & M_3
\end{bmatrix}, C=\begin{bmatrix}
c_1+c_2 & -c_2 &0 \\
-c_2 & c_2+c_3 & -c_3 \\
0 & -c_3 & c_3
\end{bmatrix}, K=\begin{bmatrix}
k_1+k_2 & -k_2  &0\\
-k_2 & k_2+k_3 & -k_3 \\
0 & -k_3 & k_3
\end{bmatrix}, B=\begin{bmatrix}1\\ 0 \\0 \end{bmatrix}$$
Finally, let $x_1$ be the state of the motor, $x_2,x_3,x_4$ the position of the first, second cart and third cart, and $x_5$, $x_6$, $x_7$ their velocities, then the stace space model, is:

\begin{equation}
\begin{cases}
\dot{x}=
\left[ \begin{array} { c|c|c  } 
                \begin{array}{c} 
               -\frac{R}{L} \\ 
                0 \\
                0 \\
                0
                \end{array} &
                \begin{array}{ccc} 
               0 & 0 & 0\\ 
                0 & 0  &0 \\
                0  & 0  & 0\\
                0 & 0 &0
                \end{array}&
                \begin{array}{ccc} 
               -\frac{\gamma}{L} & 0 &0\\ 
                1 & 0 &0 \\
                0  & 1  &0 \\
                0 & 0 & 1
                \end{array} \\
                \hline 
                M^{-1}B\gamma& -M^{-1}K & -M^{-1}C
\end{array} \right] 
x+\begin{bmatrix}\frac{1}{L} \\ 0 \\ 0 \\ 0 \\ 0\\0 \\0 \end{bmatrix}v(t)
\end{cases}
\end{equation}
where $x \in \mathbb{R}^7$.
%\section{bemf}
%TF from V to I considering bemf
%$$i(s) = \frac{1}{Ls+R}(v-K_e s\theta)$$
%But $\theta = \frac{2}{D}x$, and
%$$X(s) = \frac{\gamma}{Ms^2+Cs+K}i(s)$$
%Therefore:
%$$\theta = \frac{2\gamma}{D}\frac{1}{Ms^2+Cs+K}i(s)$$
%Then:
%$$i(s) = M1(s)(v-s\gamma^2\frac{1}{Ms^2+Cs+K}i(s)$$
%$$i(s) (1+\frac{s\gamma^2}{Ms^2+Cs+K} = M1(s) v$$
%$$i(s) = M1(s) \frac{Ms^2+Cs+K}{Ms^2+Cs+K+s\gamma^2}v$$