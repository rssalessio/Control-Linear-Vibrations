
\part{System modelling}

\section{Motor, Pinion and Rack modelling} 
A DC Brushless motor can be modelled with a simple low-pass filter transfer function. Because of that, only a resistance $R$ and an inductance $L$ are needed to model it. Thus, let $D$ be the diameter of the disk attached to the motor, $\theta$ the the angular position of the disk, $J$ the inertia of the motor, and $c_{l}(t)$ the load torque. We can also assume non-linearities based on the angle and its rate $\dot{\theta}$:  we will denote such non-linearities with $f_{m}(\theta,\dot{\theta})$.\\ \\
 The output torque of the motor, then, is  given by:
$$c(s) = \frac{K_{e}}{Ls+R} (v(s)-K_{e}s \theta(s))$$
where $v(s)$ is the Laplace transform of $v(t)$, the input voltage to the motor.\\Notice that the term $K_{e}s \theta(s)$ describes the back-emf effect, though, from experimental results as described \ref{olvscl}, it's not included in our models. Therefore the output current of the motor, is :
$$i(s) = \frac{1}{Ls+R} v(s)$$
which can be described with a state space model:
$$\begin{cases} \dot{x} =-\frac{R}{L}x + v(t)\\ i = \frac{1}{L}x\end{cases}$$
The differential equation describing the motion of the disk is:
$$J\ddot{\theta}=c(t)-c_{l}(t) -f_{m}(\theta,\dot{\theta})$$
In our case, since it's not possible to detach the pinion and the rack from the motor, $J$ includes the inertia of that system, and $c_{l}(t)$ is the torque load attached to the rack. 
\\ \\
The rack position is given by $x$, which is equal to $0$ when the rack is at the center. Since the rack is attached to the disk, and by neglecting non-lineart effects such as back-lash, we can say that $x=\frac{D}{2}\theta $. Moreover,  the total force transmitted from the motor is:
$$F(t) =c_l (t) =  \frac{2}{D} (c(t)-J\ddot{\theta}) = \gamma i(t) - \frac{4}{D^2}J \ddot{x}$$
Where $\gamma = \frac{2K_e}{D}$.

\section{Carts, Springs and Dampers Modelling}
Each cart has the same shape and mass $M_{c}$. Let $M_{i}$ denote the total mass of the $i$-eth cart including the load, $i \in {1,2,3}$. \\
Let $x_{i}$ be the position of each cart, in \SI{}{\cm}. The small friction coefficient of the sliding guide can be approximated with a viscous friction $C_{s}(M)$ which depends on the mass of the cart plus the load. \\ \\
Finally, each spring is modeled as linear spring. Since we have 3 springs, we have labeled their stiffness as $K_l, K_m, K_h$ where $l,m,h$ stand for low, medium and high. The damping contribute given by each spring is labeled as $C_l, C_m, C_h$. Therefore let $C_{i}$ denote the total damping contribution for the $i$-eth cart.\\ \\
Therefore, because of Newton's First Law, each cart has equation:
$$M_i \ddot{x}_i + C_{i} \dot{x}_i + K_{i}x_{i}=F(t)$$
Thus, for each cart 2 states are needed to describe its behaviour.\\ \\
\section{1 DOF State Space Model}
Including the motor 3 states are needed to describe the behaviour of the system. Let $x_{1}$ be the state of the motor, $x_{2}$ the position of the cart and $x_{3}$ its velocity. The measurable state are the first one and the third one, but the first one needs to be divided by the inductance $L$ as described in the previous section. Therefore the equations to consider are:
\begin{equation}
\begin{cases}
\dot{x}_1 = -\frac{R}{L}x_{1}+v(t), \quad i(t) = \frac{1}{L}x_1 \\
\dot{x}_2 = x_3 \\
M\dot{x}_3 = -Cx_3 -K x_2 + F(t)
\end{cases}
\end{equation}
Since:
$$F(t)=\gamma i(t) - \frac{4}{D^2}J \ddot{x}$$
By letting $\hat{M} = (M+\frac{4}{D^2}J)$, we have:

\begin{equation}
\begin{cases}
\dot{x}=\begin{bmatrix}
-\frac{R}{L} &0 & 0 \\
0 & 0 & 1 \\ 
 \frac{\gamma}{ML} & -\frac{K}{\hat{M}} & -\frac{C}{\hat{M}}
\end{bmatrix}
+
\begin{bmatrix}1 \\ 0 \\ 0\end{bmatrix} v(t) \\
y = \begin{bmatrix} \frac{1}{L} & 0 & 0 \\ 0 & 1 & 0 \end{bmatrix}x
\end{cases}
\end{equation}
where $v(t)$ is the external control input.
\section{2 DOF State Space Model}
To derive the equations of motion for the carts we can use the Lagrangian approach.\\ Let $T,V,D$ be the kinetic, potential and dissipated energy. Then:
\begin{align*}
T  &= \frac{1}{2} \Big(M_1 + \frac{4}{D^2}J \Big) \dot{x_1}^2 + \frac{1}{2}M_2 \dot{x_2}^2 \\
V &= \frac{1}{2}k_1x_1^2 + \frac{1}{2}k_2(x_2-x_1)^2 \\
D &= \frac{1}{2}c_1\dot{x_1}^2 + \frac{1}{2}c_2(\dot{x_2}-\dot{x_1})^2
\end{align*}
Let $Q$ be the external forces acting on the systems:
\begin{align*}
Q =\begin{bmatrix} 1 \\ 0 \end{bmatrix} \gamma i(t)
\end{align*}
The equations of motion are given by:
$$\frac{d}{dt}\Big(\frac{\partial T}{\partial x_i} \Big) -\frac{\partial T}{\partial \dot{x}_i} + \frac{\partial V}{\partial x_i} + \frac{\partial D}{\partial \dot{x}_i} = Q_i$$
Thus we can write:
$$
\begin{bmatrix}
\hat{M}_1 & 0 \\
0 & M_2
\end{bmatrix}
\ddot{x} + 
\begin{bmatrix}
c_1+c_2 & -c_2 \\
-c_2 & c_2
\end{bmatrix}
\dot{x}+
\begin{bmatrix}
k_1+k_2 & -k_2 \\
-k_2 & k_2
\end{bmatrix}
x = \begin{bmatrix} 1 \\ 0 \end{bmatrix}\gamma i(t)$$
Then:
$$\ddot{x} = M^{-1}C\dot{x}+M^{-1}Kx+ M^{-1}B\gamma i(t)$$
where:
$$M=\begin{bmatrix}
\hat{M}_1 & 0 \\
0 && M_2
\end{bmatrix}, C=\begin{bmatrix}
c_1+c_2 & -c_2 \\
-c_2 & c_2
\end{bmatrix}, K=\begin{bmatrix}
k_1+k_2 & -k_2 \\
-k_2 & k_2
\end{bmatrix}, B=\begin{bmatrix}1\\ 0 \end{bmatrix}$$
Finally, let $x_1$ be the state of the motor, $x_2$ and $x_3$ the position of the first and second cart, and $x_4$, $x_5$ their velocities, then the stace space model, is:

\begin{equation}
\begin{cases}
\dot{x}=
\left[ \begin{array} { c|c|c  } 
                \begin{array}{c} 
               -\frac{R}{L} \\ 
                0 \\
                0 
                \end{array} &
                \begin{array}{cc} 
               0 & 0 \\ 
                0 & 0 \\
                0  & 0 
                \end{array}&
                \begin{array}{cc} 
               0 & 0 \\ 
                1 & 0 \\
                0  & 1 
                \end{array} \\
                \hline 
                M^{-1}B\frac{\gamma} {L}& -M^{-1}K & -M^{-1}C
\end{array} \right] 
x+\begin{bmatrix}1 \\ 0 \\ 0 \\ 0 \\ 0\end{bmatrix}v(t)\\
y(t) = \begin{bmatrix} \frac{1}{L} & 0 & 0 & 0 & 0 \\
0 & 1 & 0 & 0 & 0 \\
0 & 0 & 1 & 0 & 0 \end{bmatrix}x
\end{cases}
\end{equation}
where $x \in \mathbb{R}^5$.

\section{3 DOF State Space Model}
As for 2 degree of freedom we can make use again of the Lagrangian Approach:
\begin{align*}
T &= \frac{1}{2} (M_1 + \frac{4}{D^2}J)\dot{x}_1^2+\frac{1}{2}M_2\dot{x}_2^2 + \frac{1}{2}M_3 \dot{x}_3^2 \\
V &= \frac{1}{2}k_1x_1^2 + \frac{1}{2}k_2(x_2-x_1)^2 +\frac{1}{2}k_3 (x_3-x_2)^2\\
D &= \frac{1}{2}c_1\dot{x_1}^2 + \frac{1}{2}c_2(\dot{x_2}-\dot{x_1})^2 +\frac{1}{2}c_3(\dot{x}_3-\dot{x}_2)^2
\end{align*}

Let $Q$ be the external forces acting on the systems:
\begin{align*}
Q =\begin{bmatrix} 1 \\ 0 \\ 0 \end{bmatrix} \gamma i(t)
\end{align*}
The equations of motion are given by:
$$\frac{d}{dt}\Big(\frac{\partial T}{\partial x_i} \Big) -\frac{\partial T}{\partial \dot{x}_i} + \frac{\partial V}{\partial x_i} + \frac{\partial D}{\partial \dot{x}_i} = Q_i$$
Thus we can write:
$$
\begin{bmatrix}
\hat{M}_1 & 0 & 0 \\
0 & M_2 & 0 \\
0 & 0 & M_3
\end{bmatrix}
\ddot{x} + 
\begin{bmatrix}
c_1+c_2 & -c_2 &0 \\
-c_2 & c_2+c_3 & -c_3 \\
0 & -c_3 & c_3
\end{bmatrix}
\dot{x}+
\begin{bmatrix}
k_1+k_2 & -k_2  &0\\
-k_2 & k_2+k_3 & -k_3 \\
0 & -k_3 & k_3
\end{bmatrix}
x = \begin{bmatrix} 1 \\ 0 \\ 0\end{bmatrix}\gamma i(t)$$
Then:
$$\ddot{x} = M^{-1}C\dot{x}+M^{-1}Kx+ M^{-1}B\gamma i(t)$$
where:
$$M=\begin{bmatrix}
\hat{M}_1 & 0 & 0 \\
0 & M_2 & 0 \\
0 & 0 & M_3
\end{bmatrix}, C=\begin{bmatrix}
c_1+c_2 & -c_2 &0 \\
-c_2 & c_2+c_3 & -c_3 \\
0 & -c_3 & c_3
\end{bmatrix}, K=\begin{bmatrix}
k_1+k_2 & -k_2  &0\\
-k_2 & k_2+k_3 & -k_3 \\
0 & -k_3 & k_3
\end{bmatrix}, B=\begin{bmatrix}1\\ 0 \\0 \end{bmatrix}$$
Finally, let $x_1$ be the state of the motor, $x_2,x_3,x_4$ the position of the first, second cart and third cart, and $x_5$, $x_6$, $x_7$ their velocities, then the stace space model, is:

\begin{equation}
\begin{cases}
\dot{x}=
\left[ \begin{array} { c|c|c  } 
                \begin{array}{c} 
               -\frac{R}{L} \\ 
                0 \\
                0 \\
                0
                \end{array} &
                \begin{array}{ccc} 
               0 & 0 & 0\\ 
                0 & 0  &0 \\
                0  & 0  & 0\\
                0 & 0 &0
                \end{array}&
                \begin{array}{ccc} 
               0 & 0 &0\\ 
                1 & 0 &0 \\
                0  & 1  &0 \\
                0 & 0 & 1
                \end{array} \\
                \hline 
                M^{-1}B\frac{\gamma} {L}& -M^{-1}K & -M^{-1}C
\end{array} \right] 
x+\begin{bmatrix}1 \\ 0 \\ 0 \\ 0 \\ 0\\0 \\0 \end{bmatrix}v(t)\\
y(t) = \begin{bmatrix} \frac{1}{L} & 0 & 0 & 0 & 0 & 0 &0\\
0 & 1 & 0 & 0 & 0 & 0 &0 \\
0 & 0 & 1 & 0 & 0 &0  &0 \\
0 & 0 & 0 & 1 & 0 & 0 &0 \end{bmatrix}x
\end{cases}
\end{equation}
where $x \in \mathbb{R}^7$.
